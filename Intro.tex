\documentclass[11pt,a4paper]{article}
\usepackage{natbib}

\linespread{1.5}

\begin{document}

The emergence of Global Value Chains (GVCs) offers a new path to industrialisation for developing countries. As \cite{riba12} phrases it, internationally fragmented production allows developing countries to join existing supply chains instead of building them. This brings about many potential advantages for these countries. Connecting with firms from advanced nations allows developing nations, for instance, to benefit from their sophisticated technologies and know-how. In addition, relying on an existing production network frees them from constraints imposed by economies of scale and the increased specialisation, that GVCs imply, limits the negative impact of unproductive parts of the domestic supply chain. After all, when competition moves from goods to tasks, comparative advantage becomes much finer and does not require a broad range of productive stages domestically. Conditional evidence for such a positive impact of GVC participation in low- and middle-income countries is presented in \citet{viku15} and \citet{unct13}.

The quick expansion of GVCs has been documented in recent work. For instance, \citet{dahuetal98, dahuetal01} show in two early seminal contributions that GVCs are responsible for a major share of the total growth in world trade from 1970 to 1990. \citet{rojoguno12} and \citet{ribajalo13} find that this growth in GVC trade has even accelerated in the recent two decades. Furthermore, this work has not only revealed a rapid rise in production fragmentaion across borders but it has also re-evaluated important indicators of trade, such as bilateral trade imbalances and revealed comparative advantage\footnote{See \citeauthor{joamsoca15} (forthcoming) for an extensive review of the literature on GVCs and outsourcing.} showing that calculating value added trade indicators is central to a better understand of countries' trade patters and competitiveness. 

However, these contributions typically have two shortcomings. Firstly, most evidence is based on data from high-income countries. The reason is that reliable time-series of both national and international input-output tables have only been available for these countries. In addition, the evidence is regularly based on a small sample of GVC indicators that hide valuable information stemming from more decomposed and disaggregated indicators. 

In this paper we address these issues by applying a novel and more detailed gross export decomposition developed by \citet{zhwaetal13} to a new set of Inter-Country Input-Output tables (ICIOs) with extensive country coverage provided by the OECD. The new ICIOs allow us to get a better understanding of the GVC activities of low- and middle-income countries while the new decomposition allows us to zoom in more closely at these activities revealing information not available from standard GVC indicators.

\end{document}
